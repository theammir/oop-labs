%! TeX program = lualatex
\documentclass[a4paper,14pt]{extarticle}
\usepackage[left=2.5cm,right=1.5cm,
		top=2cm,bottom=2cm,bindingoffset=0cm]{geometry}
\usepackage{fontspec}
\usepackage{fancyhdr}
\usepackage{fancyvrb}
\setmainfont{FreeSerif}
\setmonofont{FiraCode}
\usepackage[english,ukrainian]{babel}
\usepackage[position=bottom]{caption}

\usepackage{minted}
\setminted{style=arduino}
\setminted{mathescape}

\usepackage{caption}
\captionsetup[listing]{name=Файл}

\pagestyle{fancy}
\fancyhead{}
\fancyfoot[C]{}
\fancyfoot[R]{\thepage}
\setlength{\headheight}{17pt}
\renewcommand{\headrulewidth}{0pt}

\setcounter{secnumdepth}0

\newcommand{\shell}[2][build/temp.tex]{%
	\immediate\write18{#2 > #1}%
	\CatchFileDef{\out}{#1}{\endlinechar=13}%
}

\newcommand{\codefile}[2]{%
	\begin{center}
		\inputminted[tabsize=2, breaklines, fontsize=\small]{#1}{\basedir/\thelabid/#2}
		\captionof{listing}{\detokenize{#2}}
	\end{center}
}

\newcommand{\thetitlepage}[4]{
	\def\thelabno{#1}
	\def\thelabid{\thelabno}
	\def\thestudentno{#4}
	
	\begin{titlepage}
	\begin{center}
		\vspace{1cm}
		\textbf{Міністерство освіти і науки України\\
		Національний технічний університет України\\
		<<Київський політехнічний інститут імені Ігоря Сікорського>>\\
		Факультет інформатики та обчислювальної техніки\\
		Кафедра обчислювальної техніки\\}
		\vspace*{5cm}
		\textbf{Лабораторна робота №#1\\}
		\vspace{0.5cm}
		з дисціпліни\\
		<<Об'єктно-орієнтоване програмування>>
		\vspace*{7cm}
		
		\begin{tabular}{ll}
		\begin{minipage}[t]{0.54\linewidth}
			Виконав:\\
			студент групи #2\\
			#3\\
			номер у списку групи: \thestudentno\par
		\end{minipage}
		&
		\begin{minipage}[t]{0.3\linewidth}
			Перевірив:\\
			Порєв В.~М.
		\end{minipage}
		\end{tabular}

		\vfill
		Київ \the\year\\
	\end{center}
	\end{titlepage}
	\setcounter{page}2
}

\newcommand{\taskdesc}{\section*{Завдання}}

\newcommand{\taskspec}{\section{Варіант \thestudentno}}

\newcommand{\codetext}{\section{Текст програм}}

\newcommand{\tasktest}{\section{Результати тестування програми}}

\newcommand{\conclusion}{\section{Висновок}}

% vim: ts=2: sw=2
